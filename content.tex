\section{Über das FIfF, den µc³, unsere Cryptoparty, die Anderen, Alice und Bob}
  \begin{frame}{Das FIfF}
    Forum InformatikerInnen für Frieden und gesellschaftliche Verantwortung e.V.
  \end{frame}
  \begin{frame}{Der µc³ - Was tun wir?}
    \begin{itemize}
      \item{Kreativer Umgang mit Technik...}
      \item{... und deren Auswirkungen auf die Gesellschaft}
      \item{soll sagen: Auf Chancen hinweisen, vor Risiken warnen}
      % FIXME - Tilde wird gefressen
      \item{Viele Leute (~129 Mitglieder), viele Themen}
      \item{Für: Informations- und Kommunikationsfreiheit}
      \item{Gegen: Zensur}
      \item{Kunst, Kultur und Wissenschaft fördern}
      \item{Immer: Spaß am Gerät}
    \end{itemize}
  \end{frame}
  \begin{frame}{Der µc³ - Wer sind wir?}
    \begin{itemize}
      \item{Eine galaktische Gemeinschaft von Lebewesen,}
      \item{unabhängig von Alter,}
      \item{Geschlecht,}
      \item{Rasse}
      \item{sowie gesellschaftlicher Stellung}
    \end{itemize}
  \end{frame}
  \begin{frame}{Die Cryptoparty}
    \begin{itemize}
      \item{Wir zeigen euch wie ihr Verschlüsselungstechniken für euch nutzen könnt}
      \begin{itemize}
         \item{GnuPG - für E-Mail-Verschlüsselung und -Signierung}
         \item{OTR - für Chatverschlüsselung}
         \item{Truecrypt - für Datenträgerverschlüsselung}
         \item{Keepass - für sichere Passwortspeicherung}
         \item{TOR - Anonymeres surfen im WWW}
      \end{itemize}
      \item{Wir versuchen auch auf Risiken und Probleme dabei hinzuweisen}
    \end{itemize}
  \end{frame}
  \begin{frame}{Die Anderen}
    \begin{itemize}
      \item{Bei Fragen: Kai, Sylvia, Andi, W, ...}
      \item{WLAN SSID: diskordia, diskordia2.4}
      \item{WLAN Key: haileris}
      \item{Achtung! Ihr seid im Internet.}
    \end{itemize}
  \end{frame}
  \begin{frame}{Alice und Bob}
    \begin{itemize}
      \item{Fiktive Figuren, bekannt aus Film und Fernsehen (na gut, der fachbezogenen Literatur}
      \item{Alice - Sender, will eine Nachricht verschlüsselt verschicken}
      \item{... meistens an Bob}
      \item{Bob - Empfänger, will die verschlüsselte Nachricht von Alice (rechtmässig) lesen}
      \item{Carol und Dave, sind Dritte, aber lieb}
      \item{Eve, von engl. eavesdropper, ist eine passive Lauscherin}
      \item{Mallory, Marvin oder Mallet, von engl. malicious (hinterhältig, heimtückisch), sind aktive Angreifer}
      \item{... und noch viele mehr}
    \end{itemize}
  \end{frame}
\section{Crypto - was ist das und was ist es nicht?}
% Schutz vor kleiner Schwester vs. Schutz vor Geheimdiensten, Industriespionage und bösen Regimen
  \begin{frame}{Die Wissenschaft Kryptographie}
    \begin{itemize}
%      \item{stammt vom altgr. κρυπτός (geheim, verborgen) und γράφειν (schreiben) ab}
      \item{ursprünglich Wissenschaft der Verschlüsselung von Informationen}
      \item{heute: Teilbereich der Kryptologie}
      \item{zweiter Teil: Kryptoanalyse - Methoden und Techniken um Kryptographie zu brechen/wiederlegen)}
    \end{itemize}
  \end{frame}
  \begin{frame}{Wofür nun Kryptographie?}
    \begin{itemize}
      \item{Nicht zu verwechseln mit der Steganographie (bedeckt schreiben) (UV-Stift)} 
      \item{Schützt nicht (nur) das Tagebuch vor der kleinen Schwester,}
      \item{sondern kriegsentscheidende(!) Informationen vor'm Feind.}
      \item{Enormes kommerzielles Interesse (Industriespionage)}
      \item{Aber auch militärisch (Kommunikation im Krieg, diplomatische Informationen)}
      \item{Und geheimdienstlich (sog. Terrorabwehr)}
      \item{Heute kann (fast) jeder Kryptographie nutzen, YEAH! Aber warum?}
    \end{itemize}
  \end{frame}
\section{Eine kurze Reise in die Vergangenheit}
% http://de.wikipedia.org/wiki/Geschichte_der_Kryptographie
  \begin{frame}{Die ersten Schritte}
    \begin{itemize}
      \item{drittes Jahrtausend v. Chr.: altägyptische Kryptographie}
      \item{Andere Schriftzeichen, die nur eingeweihte kannten}
      \item{fünftes Jahrhundert v. Chr.: Griechen nutzen Skytala (Verschlüsselungsstab) (Transposition)}
      \item{100 v. Chr. - 44 n. Chr.: Cäsar-Chiffre}
      \item{Im Mittelalter 500 - 1500 quasi nichts Neues}
      \item{Weglassen von Vokalen, andere Alphabete, ... Lame!}
    \end{itemize}
  \end{frame}
  \begin{frame}{Krieg, tolle Motivation...}
    \begin{itemize}
      \item{1466 - in Italien verfeindete Stadtstaaten}
      \item{Aus Angst vor Mitlesern Bedarf an Verschlüsselung}
      \item{Leon battista Alberti beschreibt die Chiffrierscheibe}
      \item{Wieder sehr ähnlich zum altgriechischen Ansatz}
    \end{itemize}
  \end{frame}
  \begin{frame}{Vigenère-Chiffre}
    \begin{itemize}
      \item{Blaise de Vigenère 'klaut' 1586 die Tabula recta Chiffre}
      % wie funktioniert die?
      \item{Eigentlich von einem Benediktinerabt Johannes Trithemius 1508 niedergeschrieben}
      \item{300 Jahre lang ungebrochen!}
    \end{itemize}
  \end{frame}
  \begin{frame}{Erster Weltkrieg, ADFGX}
    \begin{itemize}
      \item{Telegraphie}
      \item{Ab 1. März 1918 an der deutschen Westfront eingesetzt}
      \item{Substitution (Ersetzung von Zeichen durch andere)}
      \item{gefolgt von einer Transposition (Vertauschung der Anordnung der Zeichen)}
      \item{Im April (1918 zu Beginn deutscher Frühjahrsoffensive) geknackt}
      \item{durch franz. Artillerie-Offizier Capitaine Georges Painvin, vermutlich dadurch Einnahme von Paris verhindert}
      \item{mehrere Iterationen, aber nie lange gehalten}
    \end{itemize}
  \end{frame}
  \begin{frame}{Zweiter Weltkrieg - Maschinen übernehmen}
    \begin{itemize}
      \item{Nach den Problemen im 1. Weltkrieg direkt Entwicklung erster Maschinen}
      \item{Gilbert Vernam 1918: Idee der One-Time-Pads}
      \item{Franzosen verwendeten im zweiten Weltkrieg ein schwer erhältliches Buch als OTP}
      \item{Deutsche haben das Buch bekkommen und haben nachträglich alles entschlüsselt}
      \item{Enigma: Sehr populäre Rotor-Chiffriermaschine -> polyalphabetische Verschlüsselung}
      \item{Enigma: Idee von Arthur Scherbius, Februar 1918, ab den 1930er populär, 30k-200k Maschinen}
      \item{Enigma: 1937 von Polen kurz geknackt, 1938 mehr Walzen(Rotoren) -> wieder abgehängt}
      \item{Kryha-Maschine vom Ukrainer Alexander von Kryha: 1933 geknackt, bis 1950 genutzt!}
      \item{Viele Rotor-Maschinen, die meisten bald geknackt}
    \end{itemize}
  \end{frame}
  \begin{frame}{Und dann?}
    \begin{itemize}
      \item{Militärische Verfahren immer schlechter nachvollziehbar, wegen Gehimhaltung}
      \item{Computer, Satelliten, neue Kommunikationswege -> abhörbar}
      \item{Industrie fing an, interessiert zu sein}
      \item{Forschungsgelder waren verfügbar}
      \item{Wissenschaftliches Arbeiten etablierte sich}
      \begin{itemize}
        \item{1. These aufstellen, Kryptographie}
        \item{2. Veröffentlichen, Magazine, Open Source, Wettbewerbe}
        \item{3. Versuch zu Widerlegen, Kryptoanalyse}
      \end{itemize}
      \item{Resulat: Immer bessere, offen verfügbare Algorithmen}
      \item{Viele kommerzielle Versuche, zu wenig Kryptoanalyse, immer wieder Fehlschläge}
      \item{Security by Obscurity funktioniert hier nicht!}
      \item{Willkommen im Heute!}
    \end{itemize}
  \end{frame}
\section{Symmetrische Verschlüsselung}
  \begin{frame}{Symmetrische (private key) Verschlüsselung - WTF?}
    \begin{itemize}
      \item{Viele Varianten, eine Gemeinsamkeit:}
      \item{Gleicher Schlüssel für's Ver- und Ent-schlüsseln}
      \item{Einfaches Beispiel: Caesar-Chiffre}
    \end{itemize}
  \end{frame}
  \begin{frame}{Caesar-Chiffre}
    \begin{itemize}
      \item{Npu jvr thg qnff avrznaq jrvff qnff vpu Ehzcryfgvympura urvff}
      \item{Hirntauglich: A = N, B = O, C = P, ...}
      \item{Computertauglich:}
      \item{$\text{encrypt}_K(P) = (P + K) \mod{26}$}
      \item{$\text{decrypt}_K(C) = (C - K) \mod{26}$}
      \item{Mal ausprobieren? - Belohnung: Mate!}
    \end{itemize}
  \end{frame}
  \begin{frame}{}
    \begin{itemize}
      \item{Problem: Die Kryptoanalytiker lachen darüber nur schlecht}
      \item{Häufigkeit von Buchstaben erlauben z.B. Rückschlüsse auf Key!}
      \item{Das ist aber nur eine sehr einfache von vielen Methoden}
      \item{Antwort: Die Kryptographen legen nach..}
      \item{RC4, Blowfish, DES, 3DES, Twofisch, ... wirklich lange Liste}
      \item{Block- vs. Streamchiffre, deutlich mehr als nur Substitution}
    \end{itemize}
  \end{frame}
  \begin{frame}{Symmetrische Verschlüsselung - Zusammengefasst}
    \begin{itemize}
      \item{Gleicher Schlüssel für Ver- und Ent-schlüsseln}
      \item{Selbst recht sichere Implementierungen sehr schnell}
      \item{Dadurch geeignet größere Datenmengen zu Verschlüsseln}
      \item{Aaaaber: Beide müssen den Schlüssel kennen, das ist nicht sehr praktisch}
      \item{Deshalb: Asymmetrische Verschlüsselung}
    \end{itemize}
  \end{frame}
\section{Asymmetrische (public key) Verschlüsselung}
% Asymmetrische Verschlüsselung (rsa, ..?)
  \begin{frame}{Asymmetrische (public key) Verschlüsselung}
    \begin{itemize}
      \item{Gemeinsammer Schlüssel bringt Probleme. Lösung:}
      \item{Ein Schlüssel für's Verschlüssen: Public Key}
      \item{Kann öffentlich sein!}
      \item{Ein Schlüssel für's Entschlüssen: Private Key}
      \item{Muss unbedingt privat bleiben, da man damit alle Nachrichten entschlüsseln kann}
      \item{Basieren auf 'mathematischen Probleme', z.B. Primzahlen, ...}
    \end{itemize}
  \end{frame}
% Summup: Private (decrypt) und Public (crypt) key, +sender braucht meinen private key nicht, -sehr resourcenfressend, -eher für kleine nachrichten geeignet -> hybrid, known secret
  \begin{frame}{Public Key Verschlüsselung in der Praxis}
    \begin{itemize}
      \item{Kann zum Verschlüsseln an 'Fremde' Partner dienen}
      \item{Kann zum Signieren der eigenen Nachrichten genutzt werden}
      \item{Aber nur wenn mein Public Key bekannt und 'vertrauenswürdig' ist}
      \item{Stellt die Basis für SSL/TLS dar}
      \item{Vetrauenswürdigkeit wird über sog. CA (Certificate Authorities) hergestellt}
      \item{Stellt die Basis für PGP/GPG dar}
      \item{Vertrauenswürdigkeit wird über eigene Einschätzung und 'Web of trust' hergestellt}
    \end{itemize}
  \end{frame}
  \begin{frame}{Public Key Verschlüsselung}
    \begin{itemize}
      \item{Eigener Schlüssel für's Ver- und Entschlüssen}
      \item{Der Schlüssel für's Entschlüsseln muss geheim bleiben}
      \item{Leider resourcenfressend}
      \item{Also nur für kleine Nachrichten geeignet}
      \item{Deshalb: Hybride Verschlüsselung}
    \end{itemize}
  \end{frame}
\section{Hybride Verschlüsselung}
  \begin{frame}{Hybride Verschlüsselung}
    \begin{itemize}
      \item{Wir wollen schnell Verlüsseln: Symmetrische Verschlüsselung}
      \item{Das erfordert aber ein gemeinsames Geheimnis (shared key)}
      \item{Public Key (asymmetrische Verschlüsselung) zum Austausch nutzen}
      \item{Guter Kompromiss!}
      \item{Beispiel: Diffie Hellmann Key Exchange, wichtige Basis}
    \end{itemize}
  \end{frame}
\section{Fragen, Feedback}
  \begin{frame}{Fragen, Feedback, ...}
    \begin{itemize}
      \item{Her damit!}
    \end{itemize}
  \end{frame}

\section{Messenger}
\begin{frame}{Motivation}
\begin{itemize}
\item Komfortabel, auf Smartphone einfach nutzbar
\item Wird im privaten Umfeld meist häufiger eingesetzt\\ als E-Mail
\end{itemize}

\pause
\begin{block}{Bestandsaufnahme}
Wer benutzt
\begin{itemize}
\item<+-> WhatsApp
\item<+-> Telegram
\item<+-> Threema
\item<+-> Signal
\item<+-> Jabber
\end{itemize}
\end{block}
\end{frame}

\begin{frame}{Überblick}
  \begin{itemize}
    \item Geschlossene Systeme: WhatsApp \& Co
      \begin{itemize}
        \item App-Entwickler\\
          ist Dienstanbieter\\
          und Herr über die ``technische Sprache'' (das Protokoll)
        \item Auswahl eines Dienstes\\bestimmt erreichbaren Personenkreis
        \item meist Closed Source
      \end{itemize}
    \item Offene Systeme wie Jabber/XMPP
      \begin{itemize}
        \item App-Entwickler, Dienstanbieter und Protokoll-Standardisierer sind unterschiedliche Personen
        \item App und Anbieter frei wählbar
        \item meist Open Source
      \end{itemize}
  \end{itemize}
\end{frame}

\setbeamersize{description width=1em}
\begin{frame}[t] % align on top so EFF scorecard does not move
\frametitle{Vor- und Nachteile}

\only<+>{
\begin{blex}{WhatsApp}
\item[+] Nutzdaten werden verschlüsselt
\item[+] Android: Verwendbar ohne Google Play
\item[-] Closed Source
\item[-] Anbieter erhält eine Kopie des vollständigen Telefonbuchs\\
  (nicht nur WhatsApp-Kontakte)
\end{blex}

\textbf{Tipp:} Wer sein Telefonbuch nicht freigeben will,\\
  kann den Zugriff darauf verweigern\\
  (Android: ab Version 6).

Der Nutzungskomfort ist dadurch eingeschränkt.
}

\only<+>{
\begin{blex}{Signal}
\item[+] Nutzdaten werden verschlüsselt
\item[-] Android: Nicht verwendbar ohne Google Play
\item[-] Anbieter erhält eine Kopie des vollständigen Telefonbuchs\\
  (nicht nur Signal-Kontakte)
\end{blex}}

\only<+>{
\begin{blex}{Telegram}
\item[+] \glqq Verschlüsselte Chats\grqq\ möglich
\item[-] Anbieter erhält eine Kopie des vollständigen Telefonbuchs
\item[-] Verschlüsselung standardmäßig nicht aktiv
\item[-] \glqq Normale Chats\grqq\ werden im Klartext auf den Servern gespeichert
\end{blex}}

\only<+>{
\begin{blex}{Threema}
\item[+] Chats verschlüsselt
\item[+] komfortabler Schlüsselaustausch über QR-Code
\item[+] Synchronisation des Telefonbuchs ist optional
\item[-] Closed Source
\item    kostenpflichtig
\end{blex}}


\only<+>{
\begin{blex}{Jabber/XMPP}
\item[+] Offenes System: Anbieter und App frei wählbar
\item[+] Verschlüsselung möglich (OTR oder OMEMO)
\item[+] keine Telefonbuch-Synchronisation vorgesehen
\item[-] Crypto nicht ganz so nutzerfreundlich\\wie bei kommerziellen Anbietern\\(aber dafür sind wir ja alle hier :-)
\end{blex}
\begin{itemize}
  \item    Apps: Conversations, pidgin, gajim, \ldots 
  \item    Anbieter: Unis, Hackerspaces, CCC, jabber.org, \ldots
\end{itemize}

}
\end{frame}

\begin{frame}{Zusammenfassung}
  \begin{itemize}
    \item Wer auf bestimmten Messenger nicht verzichten kann:\\Zugriff auf Kontakte verbieten!
    \item Weitere Infos: EFF Secure Messaging Scorecard\\ {\url{https://www.eff.org/secure-messaging-scorecard}}\\
      (gerade in Überarbeitung, alte Version noch verlinkt)
  \end{itemize}
  
\end{frame}

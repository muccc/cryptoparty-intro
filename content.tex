\section{Über den FIfF, den µc³ und unsere Cryptoparty}
  \begin{frame}{Das FIfF}
    Forum InformatikerInnen für Frieden und gesellschaftliche Verantwortung e.V.
  \end{frame}
  \begin{frame}{Der µc³ - Was tun wir?}
    \begin{itemize}
      \item{Kreativer Umgang mit Technik...}
      \item{... und deren Auswirkungen auf die Gesellschaft}
      \item{soll sagen: Auf Chancen hinweisen, vor Risiken warnen}
      % FIXME - Tilde wird gefressen
      \item{Viele Leute (~130 Mitglieder), viele Themen}
      \item{Kunst, Kultur und Wissenschaft fördern}
      \item{Immer: Spaß am Gerät}
    \end{itemize}
  \end{frame}
  \begin{frame}{Der µc³ - Wer sind wir?}
    \begin{itemize}
      \item{Eine galaktische Gemeinschaft von Lebewesen,}
      \item{unabhängig von Alter,}
      \item{Geschlecht,}
      \item{Rasse}
      \item{sowie gesellschaftlicher Stellung}
    \end{itemize}
  \end{frame}
  \begin{frame}{Die Cryptoparty}
    \begin{itemize}
      \item{Wir zeigen euch wie ihr Verschlüsselungstechniken für euch nutzen könnt}
      \begin{itemize}
        \item{Passwort-Management}
        \item{Anonymes Web-Surfen}
        \item{\ldots und mehr auf Anfrage}
      \end{itemize}
    \end{itemize}
  \end{frame}
  \begin{frame}{Die Anderen}
    \begin{itemize}
      \item{Bei Fragen: Kai, Sylvia, Andi, W, ...}
      \item{WLAN SSID: diskordia, diskordia2.4}
      \item{WLAN Key: haileris}
      \item{Achtung! Ihr seid im Internet.}
    \end{itemize}
  \end{frame}

\section{``Theorie'' oder Einführende Worte}
  \begin{frame}{Warnhinweise}
    \begin{itemize}
      \item 100 \%-ige Sicherheit gibt es nicht
      \item Ziel von Absichrung ist es
      \begin{itemize}
        \item Angriffe \textit{teurer} zu machen
        \item und \textit{nicht}, sie unmöglich zu machen
      \end{itemize}
      \item vor gezielten Angriffen, hinter denen viele Ressourcen (Geld) stecken, kann man sich kaum schützen
    \end{itemize}
  \end{frame}
  \begin{frame}{Wichtige Fragen für Euch}
    \begin{itemize}
      \item \textbf{Was} soll geschützt werden?
        \begin{itemize}
          \item Anonymität
          \item Echtheit des Gegenübers (Authentizität)
          \item Vertraulichkeit der Daten
          \item \ldots
        \end{itemize}
      \item Wem vertraut Ihr?
    \end{itemize}
  \end{frame}

  \begin{frame}{Vertrauen}
    \textbf{Woher weiß man, wem oder was man vertrauen kann?}
    \begin{itemize}
      \item Kurze Antwort: weiß man \textbf{nicht}.
      \item Lange Antwort
      \begin{itemize}
        \item es gibt Fragen, die man stellen kann\ldots
        \item {\ldots}und es gibt das Bauchgefühl.
      \end{itemize}
    \end{itemize}
  \end{frame}
\begin{frame}{Welche Fragen kann man stellen?}
  \begin{itemize}
    \item \textbf{Wo} sind meine Daten?
    \begin{itemize}
      \item Auf einem Blatt Papier in meiner Schublade
      \item Auf meinem Computer: Wie gut ist die Software \textbf{überprüfbar}, die meine Daten verwaltet?
        \begin{itemize}
          \item Open Source (in menschenlesbarer Form öffentlich):\\gut überprüfbar
          \item Closed Source (nur in maschinenlesbarer Form öffentlich):\\nicht überprüfbar
        \end{itemize}
      \item In der Cloud: \textbf{Wer} betreibt einen Dienst\\und \textbf{welche Motivation} hat der Betreiber?
        \begin{itemize}
          \item Macht der Betreiber einen vertrauenswürdigen Eindruck?
          \item Womit verdient der Betreiber sein Geld?
          \item Inwieweit ist man bereit,\\ \textbf{Komfort gegen Kontrolle einzutauschen}?
        \end{itemize}
    \end{itemize}
  \end{itemize}
\end{frame}

  \begin{frame}{Meta- und Nutzdaten}
    \begin{itemize}
      \item Metadaten
      \begin{itemize}
        \item E-Mail: Wer hat wann an wen mit welcher Betreffzeile eine E-Mail geschickt
        \item Surfen: Wer hat wann welche Webseite besucht
        \item Mobilfunk: Wer hat wann von wo mit wem kommuniziert
      \end{itemize}
      \item Nutzdaten
      \begin{itemize}
        \item E-Mail: Was waren Inhalt und Anhänge?
        \item Surfen: Welche Inhalte waren auf der Webseite?
        \item Mobilfunk: Was wurde geschrieben oder gesprochen?
      \end{itemize}
    \end{itemize}
  \end{frame}

  \begin{frame}{Kryptographie auf einer Folie}
    \begin{itemize}
      \item Symmetrisch: \textit{ein} \textbf{geheimer} ``Schlüssel'' (= lange Zahl)
      \begin{itemize}
        \item braucht einen \textbf{sicheren Kanal} zur Vereinbarung eines Schlüssels zwischen zwei Kommunikationspartnern
        \item Anwendungsfälle
        \begin{itemize}
          \item Festplattenverschlüsselung
          \item Zip-Dateien mit Passwort
        \end{itemize}
      \end{itemize}
      \item Asymmetrisch: ein \textbf{geheimer} \textit{und} ein dazu passender \textbf{öffentlicher} Schlüssel bilden ein \textbf{Schlüsselpaar}
      \begin{itemize}
        \item braucht \textbf{keinen} (!!) sicheren Kanal zur Vereinbarung eines Schlüssels zwischen zwei Kommunikationspartnern
        \item Verwendung
          \begin{itemize}
            \item Verschlüsseln mit öffentlichem Schlüssel,\\Entschlüsseln mit geheimem Schlüssel
            \item \textit{Digitale Signatur} von Daten mit dem privaten Schlüssel\\als Echtheitsbestätigung
          \end{itemize}
      \end{itemize}
    \end{itemize}
  \end{frame}

\section{Passwörter}
  \begin{frame}{Passwörter}
    \Large Wer nutzt mindestens fünf Internet-Dienste, bei denen man sich anmelden muss?
  \end{frame}
  \begin{frame}{Passwörter}
    \Large Und wer hat dafür mindestens drei verschiedene Passwörter?
  \end{frame}
  \begin{frame}{Passwörter: Nützliche Tipps}
    \begin{itemize}
      \item Ideal: Jedes Passwort nur einmal verwenden
      \begin{itemize}
        \item Besonders kritisch: E-Mail-Accounts, da ``Passwort zurücksetzen''-Funktionen oft per E-Mail funktionieren
      \end{itemize}
      \item Passwort-Manager sind eine sicherer, als überall das gleiche Passwort zu verwenden
      \item Wenn das zu kompliziert ist: Passwörter leicht variieren (``salzen'')
      \begin{itemize}
        \item 1Quastenfl0sser.mai für Mail
        \item 1Quastenfl0sser.son für Social Network
        \item \ldots
      \end{itemize}
    \end{itemize}
  \end{frame}
  \begin{frame}{Passwort-Manager}
    \begin{itemize}
      \item Software zur Verwaltung von Passwörtern
      \item Datenbank wird mit einem Passwort verschlüsselt
      \item Beispiel: KeePassX
    \end{itemize}
      \includegraphics[width=\textwidth]{images/keepassx.png}
    \begin{itemize}
      \item \textbf{Wichtige Passwörter trotzdem merken!}
      \item \ldots oder zumindest auf einem Zettel aufschreiben und zuhause an einem sicheren Ort lagern
    \end{itemize}
  \end{frame}
  \begin{frame}{Aber wie soll man sich Passwörter merken?}
    \begin{center}
      \includegraphics[width=0.9\textheight]{images/password_strength.png}\\
    \end{center}
    \tiny Quelle: \url{http://xkcd.com/936/}
  \end{frame}
\section{Web-Surfen}
  \begin{frame}{Anonymes Werbsurfen}
    \begin{itemize}
      \item Anfallende Meta-Daten
      \begin{itemize}
        \item Cookies und Co (HTML5 Persistent Local Storage, Flashcookies, \ldots)
        \item Browser-Fingerabdruck
        \item IP-Adresse
      \end{itemize}
      \item Was kann man dagegen tun
      \begin{itemize}
        \item \textbf{Einfach}: Tor Browser Bundle\\ \url{https://www.torproject.org}\\[.5cm]
        \item \textbf{Ohne Spuren am PC}: Tails\\ \url{https://tails.boum.org}\\[.5cm]
        \item Alternative: Browser-Plugins
          \begin{itemize}
            \item NoScript, ClickToPlugin, AdBlockPlus, Ghostery (Achtung: Closed Source), \ldots 
          \end{itemize}
      \end{itemize}
    \end{itemize}
  \end{frame}

\section{E-Mail}
  \begin{frame}{E-Mails: Was soll geschützt werden?}
    E-Mails können
    \begin{itemize}
      \item abgehört
      \item gefälscht
    \end{itemize}
    werden. Deshalb stellen wir vor, wie man
    \begin{itemize}
      \item die Vertraulichkeit (das ``Briefgeheimnis'') umsetzt
      \begin{itemize}
        \item Verschlüsselung
      \end{itemize}
      \item die Echtheit des Gegenübers sicherstellt
      \begin{itemize}
        \item Digitale Signatur
      \end{itemize}
    \end{itemize}
  \end{frame}

  \begin{frame}{Analogie zur Vertraulichkeit von E-Mails}
    \begin{itemize}
      \item E-Mails sind ``Postkarten''
      \item diese werden in ``gläsernen Fahrzeugen'' transportiert
      \begin{itemize}
        \item ``Autobahnbetreiber'' kann alles mithören
        \item ``Post'' kann alles mithören
      \end{itemize}
      \item Bei \textbf{Transportverschlüsselung} ersetzt die ``Post'' die ``gläsernen Fahrzeuge'' durch ``undurchsichtige Fahrzeuge''
      \begin{itemize}
        \item ``Autobahnbetreiber'' kann mithören,\\welche ``Post'' mit welcher anderen ``Post'' kommuniziert
        \item ``Post'' kann alles mithören
      \end{itemize}
      \item Bei \textbf{Ende-zu-Ende-Verschlüsselung} steckt der Absender die ``Postkarte'' in einen ``Briefumschlag''
      \begin{itemize}
        \item ``Autobahnbetreiber'' kann mithören,\\wer mit wem kommuniziert
        \item ``Post'' kann mithören, wer mit wem kommuniziert
      \end{itemize}
      \item Wir stellen hier Ende-zu-Ende-Verschlüsselung vor.
    \end{itemize}
  \end{frame}

  \begin{frame}{Überprüfung der Echtheit}
\begin{itemize}
  \item Was passiert, wenn zwei Parteien kommunizieren wollen, die sich nicht kennen?\\[.5cm]
  \item Allgemein: A kann mittels Digitaler Signatur sagen:\\``Der öffentliche Schlüssel von B ist echt.''
  \item Diese Aussage kann mit dem öffentlichen Schlüssel von A überprüft werden
  \item Wer A vertraut, kann also auch B vertrauen
\end{itemize}
  \end{frame}

  \begin{frame}{S/MIME und GnuPG}
    \begin{itemize}
      \item S/MIME -- ``Hierarchischer'' Vertrauensansatz
      \begin{itemize}
        \item Es gibt ``zentrale Vertrauensinstanzen'' (Certification Authorities, CAs), denen \textbf{jeder} vertraut
        \item Diese bestätigen die Echtheit der Schlüssel von  \textit{untergeordneten} CAs
        \item \ldots eine (beliebige) CA aus der Vertrauenskette kann die Echtheit von Schlüsseln von Personen bestätigen
        \item wird hier \textit{nicht} behandelt
      \end{itemize}
      \item GnuPG
      \begin{itemize}
        \item jeder kann festlegen, wem er vertraut
        \begin{itemize}
          \item er kann die Echtheit eines Schlüssels z.B. bei einem persönlichen Treffen überprüfen
        \end{itemize}
        \item jeder \textit{kann} sein Vertrauensnetz veröffentlichen (Web-of-Trust)
        \begin{itemize}
          \item Vorteil: Man kann auch ``Freunden von Freunden'' vertrauen
          \item Nachteil: Beziehungen zwischen Menschen sind öffentlich 
        \end{itemize}
        \item wird hier behandelt
      \end{itemize}
    \end{itemize}
  \end{frame}

  \begin{frame}{E-Mail-Absicherung mit GnuPG}
    \begin{centering}
      \Huge Live-Demo
    \end{centering}
  \end{frame}

\section{Fragen, Feedback}
  \begin{frame}{Fragen, Feedback, ...}
    \begin{itemize}
      \item{Her damit!}
    \end{itemize}
  \end{frame}
